\subsection{Opgaver}

Dette delkapitel vil indeholde en række opgaver til at træne den viden i har lært fra kapitel 1.

\subsubsection*{Opgave 1: Addition af brøker med ens nævnere}
Beregn $\frac{2}{5} + \frac{3}{5}$

\subsubsection*{Opgave 2: Addition af brøker med ens nævnere}
Beregn $\frac{5}{7} + \frac{2}{7}$

\subsubsection*{Opgave 3: Addition af brøker med forskellige nævnere}
Beregn $\frac{3}{4} + \frac{4}{6}$

\subsubsection*{Opgave 4: Addition af brøker med forskellige nævnere}
Beregn $\frac{4}{5} + \frac{3}{7}$

\subsubsection*{Opgave 5: Subtraktion af brøker med ens nævnere}
Beregn $\frac{3}{4} - \frac{2}{4}$

\subsubsection*{Opgave 6: Subtraktion af brøker med ens nævnere}
Beregn $\frac{4}{6} - \frac{5}{6}$

\subsubsection*{Opgave 7: Subtraktion af brøker med forskellige nævnere}
Beregn $\frac{4}{5} - \frac{3}{2}$

\subsubsection*{Opgave 8: Subtraktion af brøker med forskellige nævnere}
Beregn $\frac{5}{7} - \frac{3}{5}$

\subsubsection*{Opgave 9: Multiplikation af konstant og brøk}
Beregn $4 \cdot \frac{3}{5}$

\subsubsection*{Opgave 10: Multiplikation af konstant og brøk}
Beregn $-2 \cdot \frac{3}{5}$

\subsubsection*{Opgave 11: Multilpikation af 2 brøker}
Beregn $\frac{3}{5}\cdot \frac{4}{3}$

\subsubsection*{Opgave 12: Multiplikation af 2 brøker}
Beregn $\frac{-2}{7} \cdot \frac{3}{4}$

\subsubsection*{Opgave 13: Divison af 2 brøker}
Beregn $\frac{2}{3} : \frac{4}{5}$

\subsubsection*{Opgave 14: Divison af 2 brøker}
Beregn $\frac{4}{5} : \frac{1}{3}$

\subsubsection*{Opgave 15: Brug af kvadratsætningerne}
Ophæv parentesen $(x - 4)\cdot (x + 4)$

\subsubsection*{Opgave 16: Brug af kvadratsætningerne}
Ophæv parentesen $(x - 5)^2$

\subsubsection*{Opgave 17: Brug af kvadratsætningerne}
Ophæv parentesen $(x + 6)^2$

\subsubsection*{Opgave 18: Brug af kvadratsætningerne}
Ophæv parentesen $(x - 2)^2$

\subsubsection*{Opgave 19: Brug af kvadratsætningerne}
Ophæv parentesen $(x - 3) \cdot (x + 3)$

\subsubsection*{Opgave 20: Brug af kvadratsætningerne}
Ophæv parentesen $(x + 8)^2$



