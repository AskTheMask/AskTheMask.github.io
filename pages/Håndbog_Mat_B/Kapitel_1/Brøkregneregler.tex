\subsection{Brøkregneregler}

Det første emne vi kommer til at gennemgå er brøkregneregler. Vi skal lære addition af brøker, subtraktion af brøker, hvordan vi ganger og dividerer brøker og til sidst hvordan vi forkorter brøker. 

En brøk består af 2 tal, det øverste tal kaldes for tælleren og det nederste tal for nævneren. Når vi adderer 2 brøker gælder følgende regler

\begin{frm-thm}{Addition og subtraktion af brøker}\thlabel{add_sub}


Hvis 2 brøker har samme nævner b er additionen og subtraktionen af de 2 brøker givet ved
\[ \frac{a}{b} + \frac{c}{b} = \frac{a + c}{b}   \]
\[\frac{a}{b} - \frac{c}{b} = \frac{a - c}{b}\]

Hvis 2 brøker ikke har samme nævner er deres addition givet ved
\[ \frac{a}{b} + \frac{c}{d} = \frac{a\cdot d + c \cdot b}{b \cdot d} \]
\[\frac{a}{b} - \frac{c}{d} = \frac{a\cdot d - c\cdot b}{b \cdot d} \]
\end{frm-thm}

Vi vil nu gennemgå en række eksempler på hvordan vi laver addition og subtraktion af 2 brøker.

\subsubsection*{Eksempel 1: Addition med ens nævnere}

Vi får givet følgende brøker $\frac{2}{3}$ og $\frac{5}{3}$ og bliver bedt om at bestemme $\frac{2}{3} + \frac{5}{3}$. Da de 2 brøker har fælles nævner siger theorem \ref{add_sub} at vi skal gøre følgende
\begin{align*}
\frac{2}{3} + \frac{5}{3} = \frac{2 + 5}{3} = \frac{7}{3}
\end{align*}

Resultatet af additionen er dermed $\frac{7}{3}$.

\subsubsection*{Eksempel 2: Subtraktion med ens nævnere}

Vi får givet følgende brøker $\frac{2}{3}$ og $\frac{1}{3}$ og bliver bedt om at bestemme $\frac{2}{3} - \frac{1}{3}$. Da de 2 brøker har fælles nævnere siger theorem \ref{add_sub} at vi skal gøre følgende

\begin{align*}
\frac{2}{3} - \frac{1}{3} = \frac{2 - 1}{3} = \frac{1}{3}
\end{align*}

Resultatet af subtraktionen er dermed $\frac{1}{3}$

\subsubsection*{Eksempel 3: Addition med forskellige nævnere}

Vi får givet følgende brøker $\frac{2}{3}$ og $\frac{4}{5}$ og bliver bedt om at bestemme $\frac{2}{3} + \frac{4}{5}$. Da de 2 brøker ikke har fælles nævnere siger theorem \ref{add_sub} at vi skal gøre følgende

\begin{align*}
\frac{2}{3} + \frac{4}{5} = \frac{2\cdot 5 + 4\cdot 3}{3\cdot 5} = \frac{10 + 12}{15} = \frac{22}{15}
\end{align*}

Resultatet af additionen er dermed $\frac{22}{15}$

\subsubsection*{Eksempel 4: Subtraktion med forskellige nævnere}

Vi får givet følgende brøker $\frac{2}{3}$ og $\frac{4}{5}$ og bliver bedt om at bestemme $\frac{2}{3} - \frac{4}{5}$. Da de 2 brøker ikke har fælles nævnere siger theorem \ref{add_sub} at vi skal gøre følgende

\begin{align*}
\frac{2}{3} - \frac{4}{5} = \frac{2\cdot 5 - 4\cdot 3}{3\cdot 5} = \frac{10 - 12}{15} = \frac{-2}{15} = -\frac{2}{15}
\end{align*}


Når vi ganger et tal på en brøk, ganger 2 brøker sammen eller dividerer 2 brøker gælder følgende regler

\begin{frm-thm}{Multiplikation og division af brøker}\thlabel{mult_div}
Hvis vi ganger tallet a ind på en brøk ganger vi a ind i tælleren
\[a\cdot \frac{b}{c} = \frac{a\cdot b}{c}\]

Ganger vi 2 brøker med hinanden ganger vi deres tællere og nævnere sammen
\[\frac{a}{b} \cdot \frac{c}{d} = \frac{a\cdot c}{b\cdot d} \]

Dividerer vi 1 brøk med en anden kan vi i stedet gange med den omvendte brøk (dvs vi bytter om på tælleren og nævneren i den ene brøk)

\[\frac{a}{b} : \frac{c}{d} = \frac{a}{b} \cdot \frac{d}{c}\]
\end{frm-thm}

Vi vil nu gennemgå en række eksempler på hvordan vi kan bruge de ovenstående regler

\subsubsection*{Eksempel 5: Multiplikation af konstant og brøk}

Vi får givet konstanten $4$ og brøken $\frac{3}{5}$ og bliver bedt om at bestemme $4\cdot \frac{3}{5}$. Når vi ganger et tal på en brøk siger theorem \ref{mult_div} at vi skal gøre følgende

\begin{align*}
4\cdot \frac{3}{5} = \frac{4\cdot 3}{5} = \frac{12}{5}
\end{align*}

Resultatet af multiplikationen er dermed $\frac{12}{5}$


\subsubsection*{Eksempel 6: Multiplikation af 2 brøker}

Vi får givet de 2 brøker $\frac{2}{3}$ og $\frac{4}{5}$ og bliver bedt om at bestemme $\frac{2}{3} \cdot \frac{4}{5}$. Når vi ganger 2 brøker sammen siger theorem \ref{mult_div} at vi skal gøre følgende

\begin{align*}
\frac{2}{3} \cdot \frac{4}{5} = \frac{2\cdot 4}{3\cdot 5} = \frac{8}{15}
\end{align*}

Resultatet af multiplikationen er dermed $\frac{8}{15}$

\subsubsection*{Eksempel 7: Divison af 2 brøker}

Vi får givet de 2 brøker $\frac{2}{3}$ og $\frac{4}{5}$ og bliver bedt om at bestemme $\frac{2}{3} : \frac{4}{5}$. Når vi dividerer en brøk med en anden siger theorem \ref{mult_div} at vi skal gøre følgende

\begin{align*}
\frac{2}{3} : \frac{4}{5} = \frac{2}{3} \cdot \frac{5}{4} = \frac{2\cdot 5}{3\cdot 4} = \frac{10}{12}
\end{align*}

Resultatet af divisionen er dermed $\frac{10}{12}$


Til sidst vil vi kort snakke om hvordan man forkorter brøker. Når vi forkorter en brøk gør vi tælleren og nævneren af brøken mindre uden af ændre på selve tallet brøken repræsenterer.
Vi kigger nu på følgende eksempel.

\subsubsection*{Eksempel 8: Forkortelse af brøk}

 Hvis vi kigger på følgende brøk $\frac{2}{4}$ så repræsenterer brøkken decimaltallet 0,5. Hvis vi nu gør brøkens tæller og nævner 2 gange mindre har vi nu brøken $\frac{1}{2}$ som stadig repræsenterer decimaltallet 0,5. Vi har altså nu forkortet brøken $\frac{2}{4}$. Hvornår kan vi ikke forkorte en brøk mere og hvorfor forkorter vi brøker?