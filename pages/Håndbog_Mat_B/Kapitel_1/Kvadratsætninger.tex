\subsection{Kvadratsætninger}

I dette delkapitel vil vi snakke om de 3 kvadratsætninger, hvorfor vi bruger dem og i hvilke tilfælde man ville benytte sig af dem. De 3 kvadratsætninger er som følgende

\begin{frm-thm}{Første kvadratsætning}\thlabel{kvad_1}

Lad a og b være 1 tal og 1 bogstav eller 2 forskellige bogstaver så gælder det at

\[(a + b)^2 = a^2 + b^2 + 2ab\]
\end{frm-thm}

\begin{frm-thm}{Anden kvadratsætning}\thlabel{kvad_2}

Lad a og b være 1 tal og 1 bogstav eller 2 forskellige bogstaver så gælder det at

\[(a - b)^2 = a^2 + b^2 - 2ab\]
\end{frm-thm}

\begin{frm-thm}{Tredje kvadratsætning}\thlabel{kvad_3}

Lad a og b være 1 tal og 1 bogstav eller 2 forskellige bogstaver så gælder det at

\[(a + b)\cdot (a - b) = a^2 - b^2\]
\end{frm-thm}

For at snakke om i hvilke tilfælde vi benytter os af kvadratsætningerne er det lettest at snakke ud fra en række eksempler.

\subsubsection*{Eksempel 9: Brug af første kvadratsætning}
Hvis vi i en opgave støder på følgende parentes $(x + 3)^2$ kan vi her benytte os af første kvadratsætning til at ophøve parentesen. Vi kan bruge første kvadratsætning så længe parentesen indeholder 2 led adsklit af et plus hvor de 2 led enten består at 1 tal og 1 bogstav eller 2 forskellige bogstaver. I vores tilfælde består de 2 led af 1 bogstav $x$ og et tal $3$. Bruger vi første kvadratsætning får vi

\begin{align*}
(x + 3)^2 = x^2 + 3^2 + 2\cdot x\cdot 3 = x^2 + 9 + 6x
\end{align*}

Hvis vi ikke havde brugt første kvadratsætning kunne vi ophæve parentesen på følgende måde

\begin{align*}
(x + 3)^2 = (x + 3) \cdot (x + 3) = x^2 + 3x + 3x + 3^2 = x^2 + 6x + 9
\end{align*}

Da dette både er mere besværgeligt og tager længere tid benytter man i stedet for kvadratsætningerne.

\subsubsection*{Eksempel 10: Brug af anden kvadratsætning}
Hvis vi i en opgave støder på følgende parentes $(x - 3)^2$ kan vi her benytte os af anden kvadratsætning da de 2 led i parentesen er adskilt af et minus. Vi får

\begin{align*}
(x - 3)^2 = x^2 + 3^2 -2\cdot x\cdot 3 = x^2 + 9 - 6x
\end{align*}

\subsubsection*{Eksempel 11: Brug af tredje kvadratsætning}
Hvis vi i en opgave støder på følgende parentes $(x + 3)\cdot(x - 3)$ kan vi her benytte os af den tredje kvadratsætning da de 2 parenteser indeholder de samme led, men i den ene parentes er de adskilt af plus og i den anden af minus. Vi får

\begin{align*}
(x + 3)\cdot (x - 3) = x^2 - 3^2 = x^2 - 9
\end{align*}
