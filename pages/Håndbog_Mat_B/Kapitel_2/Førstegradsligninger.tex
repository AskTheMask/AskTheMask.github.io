\subsection{Førstegradsligninger}

I dette delkapitel vil vi gennemgå hvordan man løser en førstegradsligning. En førstegradsligning indeholder en række led som består af tal og en ukendt variabel, typisk x. Et eksempel på en førstegradsligning kunne være 
\begin{align*}
    5x = 2x + 9
\end{align*}

Når vi løser en førstegradsligning som den ovenfor betyder det at vi isolerer x. Vi skal altså have x til at stå alene på den ene side af lighedstegnet. Når vi har løst førstegradsligningen har vi fundet ud af hvilken værdi x skal have for at begge sider af lighedstegnet er ens. 
Vi vil nu se et eksempel på hvordan man løser førstegradsligningen ovenfor.

\begin{align*}
    5x &= 2x + 9\\
    \Updownarrow \hspace{1.4cm}&\\
    5x - 2x &= 2x + 9 - 2x && -2x \hspace{1mm} \text{på begge sider}\\
    \Updownarrow \hspace{1.4cm}&\\
    3x &= 9\\
    \Updownarrow \hspace{1.4cm}&\\
    \frac{3x}{3} &= \frac{9}{3} && \text{dividerer med 3 på begge sider}\\
    \Updownarrow \hspace{1.4cm}&\\
    x &= 3
\end{align*}

Løsningen til vores ligning er dermed $x = 3$. Vi kan nu kontrollere om løsningen til vores ligning faktisk er $x = 3$ ved at indsætte 3 på x's plads i ligningen og kontrollere at begge sider af lighedstegnet er ens

\begin{align*}
    5x &= 2x + 9\\
    \Updownarrow \hspace{1cm}&\\
    5\cdot 3 &= 2 \cdot 3 + 9 && \text{indsætter} \hspace{1mm} x = 3 \\
    \Updownarrow \hspace{1cm}&\\
    15 &= 6 + 9\\
    \Updownarrow \hspace{1cm}&\\
    15 &= 15
\end{align*}

Da begge sider af lighedstegnet er ens 
løser vi en ligning så finder vi den værdi der kan indsætte i vores variabel, her x, så begge sider af lighedstegnet er ens.
På den måde kan vi altid kontrollere om vi har løst vores ligning korrekt.

Hvornår løser vi ellers førstegradsligninger. Lad os betragte følgende eksempel.

Antallet af landbrug i Danmark kan for perioden 1983 - 2000 beskrives ved modellen 
\begin{align*}
    y = -2600x + 98680
\end{align*}

hvor y er antallet af landbrug, og x er antal år efter 1983.

Hvis vi nu bliver bedt om at bestemme hvor mange år der går før antallet af landbrug er faldet under en hvis grænse, fx under 40.000, så skal vi indsætte de 40.000 på y's plads, da y betegner antallet af landbrug, og isolere x, da x betegner antallet af år efter 1983. Vi kommer til at have følgende ligning

\begin{align*}
    y &= -2600x + 98680\\
    \Updownarrow \hspace{1cm}&\\
    40000 &= -2600x + 98680 && \text{Indsæt} \hspace{1mm} y = 40000
\end{align*}

Løser vi ligningen får vi

\begin{align*}
    40000 &= -2600x + 98680\\
     \Updownarrow \hspace{2.4cm}&\\
     40000 - 98680 &= -2600x + 98680 - 98680 && -98680 \hspace{1mm} \text{på begge sider}\\
      \Updownarrow \hspace{2.4cm}&\\
      -58680 &= -2600x\\
       \Updownarrow \hspace{2.4cm}&\\
       \frac{-58680}{-2600} &= \frac{-2600x}{-2600} && \text{Dividerer med -2600 på begge sider}\\
        \Updownarrow \hspace{2.4cm}&\\
        22,57 &= x\\
         \Updownarrow \hspace{2.4cm}&\\
         x &= 22,57
\end{align*}

Vi har altså fundet ud af at efter $22,57$ år er antallet af landbrug præcis 40000. Så efter $23$ år vil antallet af landbrug være faldet til under 40000. 

De 2 eksempler som blev gennemgået i dette delkapitel er typisk hvad man kan forvente at bruge førstegradsligninger til. 