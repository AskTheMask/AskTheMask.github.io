\subsection{Opgave 2}

Nedenfor er en ligning løst.
\begin{align*}
    3x + 2(x+1) + 7 &= 5\\
    3x + 2x + 2 + 7 &= 5\\
    5x + 9 &= 5\\
    5x &= -4\\
    x &= -\frac{4}{5}
\end{align*}
Forklar, hvad der er gjort i hvert trin.\\\\

\ans
I det første trin ganger vi 2 tallet ind i parentesen på hvert led således 
\begin{align*}
    2(x + 1)=2x + 2
\end{align*}
I det næste trin lægger vi ledene som indeholder x sammen og ledene som ikke indeholder x sammen.
\begin{align*}
    3x + 2x +2 +7 = 5x + 9
\end{align*}
I det næste trin trækker vi 9 fra på begge sider
\begin{align*}
    5x + 9 - 9 = 5 - 9 \Longleftrightarrow 5x = -4
\end{align*}
I det sidste trin dividerer vi med 5 
\begin{align*}
    \frac{5x}{5} = -\frac{4}{5} \Longleftrightarrow x = -\frac{4}{5}
\end{align*}
