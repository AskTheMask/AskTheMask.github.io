\subsection{Opgave 15}

Et punkt har koordinatsættet $A(8,6)$.\\\\
En linje $l$ har ligningen $y = -x+7$.\\\\
Bestem afstanden mellem $A$ og $l$.\\\\

\ans
Her bruger vi formlen for afstand mellem et punkt og en linje som siger at afstanden fra et punkt $(x_1,y_1)$ til en linje $y = ax+b$ kan bestemmes ved
\begin{align*}
    \text{dist}(P,m)=\frac{|ax_1+b-y_1|}{\sqrt{a^2 +1}}
\end{align*}
Vi indsætter tallene og får
\begin{align*}
    \text{dist}(P,m)=\frac{|(-1)\cdot 8+7-6|}{\sqrt{(-1)^2 + 1}}= \frac{|-8+1|}{\sqrt{1+1}}=\frac{|-7|}{\sqrt{2}}=\frac{7}{\sqrt{2}}\approx 4.95
\end{align*}