\subsection{Opgave 40}

På en græsplæne måles hæjden af græsset lige efter en græsslåning til 45 millimeter. Gartneren ved, at græsset vokser 5 millimeter i døgnet.

Indfør passende variable, og opstil en model, som beskriver sammenhængen mellem græssets højde og tiden siden seneste slåning.

\ans

Da vi får at vide at græsset har en starthøjde og at det vokser med samme højde hvert døgn, ved vi at 
græsset følger en lineær model/funktion. Vi ved at den generelle forskrift for en lineær funktion er 
\begin{align*}
    f(x) = ax + b
\end{align*}

Værdien a fortæller hvor meget noget stiger med og værdien b fortæller os hvad startværdien er når $x = 0$.

Da vi har med tid at gøre, vælger jeg at udskifte x med t for antal døgn, og f med h for højde. 
Da græsset vosker med 5 millimeter i døgnet må vores a værdi være $a = 5$ og da græsset lige efter en græsslåning altså når $t = 0$ er 45 millimeter må vores b værdi være $b = 45$

Modellen som beskriver sammenhængen mellem græssets højde og antal døgn siden sidste slåning bliver derfor

\begin{align*}
    h(t) = 5t + 45
\end{align*}

Hvor $h(t)$ er højden i millimeter efter t døgn, t er antal døgn, 5 er antal millimeter græsset vosker med pr døgn og 45 er antal millimeter græsset er lige efter en græsslåning.