\subsection{Opgave 41}

En gryde med vand varmes op på en kogeplade.

Lad t betegne tiden i minutter fra starten af målingerne og lad T betegne vandets temperatur. Til tiden 
$t=0$ er vandets temperatur $12^{circ}C$. Mens vandet opvarmes stiger temperaturen med $3^{\circ}C$ pr. minut.

Opstil en model for sammenhængen mellem vandes temperatur og tiden fra starten af målingerne.

\ans

Da vi får at vide at temperaturen stiger med den samme værdi for hvert minut og at vandet har en starttemperatur til tiden $t = 0$ 
har vi at gøre med en lineær model. Lineæra modeller eller lineære funktioner er på formen $f(x) = a\cdot x + b$ hvor a er hvor meget vores lineære model stiger eller falder pr x og 
b er vores startværdi der hvor $x = 0$. I vores tilfælde er $a = 3$, $b = 12$ og x er her t altså antallet minutter mens f er T altså temperaturen af vandet.
Vi får nu følgende model

\begin{align*}
    T(t) = 3\cdot t + 12
\end{align*}

$T(t)$ betegner altså vandets temperatur efter t minutter, 3 betegner antallet af grader vandet stiger med pr minut, t betegner antal minutter og 12 betegner 
vandets starttemperatur til tiden $t = 0$.


