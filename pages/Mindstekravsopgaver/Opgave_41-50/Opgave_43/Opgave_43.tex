\subsection{Opgave 43}

En funktion f er bestemt ved 

\begin{align*}
    f(x) = 3\cdot x^2
\end{align*}

Nogle værdier for f er vist i tabellen.

\begin{tabular}{|c|c|c|c|}
    \hline
    x & -1 & 0 & 2 \\\hline
    $f(x)$ &  &  & \\\hline
\end{tabular}

Udfyld tabellens tomme felter.

\ans

For at udfylde tabellens tomme felter skal vi for hver kolonnes x værdi beregne den tilsvarende værdi for $f(x)$ ved at indsætte x værdien i forskriften 
$f(x) = 3\cdot x^2$

For den første kolonne og x værdien $x = -1$ får vi den tilsvarende værdi

\begin{align*}
    f(-1) = 3\cdot (-1)^2 = 3 \cdot 1 = 3
\end{align*}

For den anden kolonne og x værdien $x = 0$ får vi den tilsvarende værdi

\begin{align*}
    f(0) = 3\cdot 0^2 = 3 \cdot 0 = 0
\end{align*}

For den første kolonne og x værdien $x = 2$ får vi den tilsvarende værdi

\begin{align*}
    f(2) = 3\cdot 2^2 = 3 \cdot 4 = 12
\end{align*}

Den udfyldte tabel kan ses nedenfor

\begin{tabular}{|c|c|c|c|}
    \hline
    x & -1 & 0 & 2 \\\hline
    $f(x)$ & 3 & 0 & 12 \\\hline
\end{tabular}

