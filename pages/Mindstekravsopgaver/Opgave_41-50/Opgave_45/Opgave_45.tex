\subsection{Opgave 45}

Tabellen viser antal frøer N i et vandhul i perioden fra 2012-2015. Tiden t måles i år fra 2012.

Følgende data er tilgængelige:

\begin{tabular}{|c|c|c|c|c|}
    \hline
    Tid t & 0 & 1 & 2 & 3 \\\hline
    Antal frøer N & 4520 & 4260 & 4000 & 3750
\end{tabular}

Det oplyses, at udviklingen i antal frøer tilnærmelsesvis kan beskrives ved en model på formen:

\begin{align*}
    N(t) = a\cdot t + b
\end{align*}

a) Bestem a og b.

\ans

Da vi får at vide at udviklingen i antal frøer kan beskrives ved $N(t) = a\cdot t + b$ ved vi at udviklingen beskrives ved en lineær model $f(x) = a\cdot x + b$

Ud fra 2 datapunkter kan vi altså beregne a værdien ved formlen

\begin{align*}
    a = \frac{y_2 - y_1}{x_2 - x_1}
\end{align*}

Jeg vælger den første kolonne (0, 4520) og den sidste kolonne (3, 3750) som mine 2 datapunkter så jeg får
$(x_1, y_1) = (0, 4520)$ og $(x_2, y_2) = (3, 3750)$. Beregner nu a værdien

\begin{align*}
    a = \frac{3750 - 4520}{3 - 0} = \frac{-770}{3} = -256,67
\end{align*}

For en lineær funktion kan vi beregne b værdien ved følgende formel

\begin{align*}
    b = y_1 - a \cdot x_1
\end{align*}

Indsætter vi tallene får vi

\begin{align*}
    b = 4520 - (-256,67) \cdot 0 = 4520
\end{align*}

Vi har nu værdierne $a = -256,67$ og $b = 4520$.

b) Hvad fortæller tallet a om udviklingen i antal frøer?

\ans

Tallet a fortæller om hvor meget antallet af frøer ændrer sig for hvert år t.
Da a værdien er $a = -256,67$ betyder det altså at antallet af frøer i vandhullet falder med ca 256 frøer om året.
