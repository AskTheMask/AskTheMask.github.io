\subsection{Opgave 47}

Funktionen f er bestemt ved forskriften 

\begin{align*}
    f(x) = 4x^3 + 5x^2 + 8
\end{align*}

Bestem f'(2).

\ans

For at bestemme f'(2) skal vi først bestemme f'(x).

Når vi bestemmer f'(x) betyder det at vi differentierer f(x).

Der gælder følgende 

\begin{align*}
    f(x) &= a\cdot x^n\\
    f'(x) &= n\cdot a \cdot x^{n-1}
\end{align*}

Når vi differentierer en konstant, a ganget på x $a\cdot x$ får vi konstanten a.

Når vi differentierer konstanter (tal) så bliver de altid 0.
I vores tilfælde består $f(x)$ af flere led så vi kan anvende de ovenstående regler på alle ledene. Vi får

\begin{align*}
    f'(x) = 3\cdot 4x^{3-1} + 2\cdot 5x^{2-1} + 0 = 12x^2 + 10x
\end{align*}

Nu har vi beregnet f'(x), og for at bestemme f'(2) indsætter vi bare 2 på x's plads i f'(x). Vi får

\begin{align*}
    f'(2) = 12\cdot 2^2 + 10\cdot 2 = 12\cdot 4 + 20 = 68
\end{align*}

Vi får altså svaret $f'(2) = 68$