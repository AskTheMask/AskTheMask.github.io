\section*{Beregning af forskrift for en lineær funktion ud fra 2 punkter}

For at bestemme forskritfen for en lineær funktion ud fra 2 punkter skal vi bruge følgende formler


\begin{frm-thm}{Hældningen af den rette linje}

Givet 2 punkter $(x_1, y_1)$ og $(x_2, y_2)$ kan hældningen, a, for den rette linje $y = ax + b$ bestemmes ved

\[a = \frac{y_2 - y_1}{x_2 - x_1}\]


\end{frm-thm}

\begin{frm-thm}{Skæringen mellem den rette linje og y-aksen}

Givet 1 punkt $(x_1, y_1)$ og hældningen a for den rette linje $y = ax + b$ kan den rette linjes skæring med y-aksen, b, bestemmes ved

\[b = y_1 - ax_1\]

\end{frm-thm}



\subsection*{Eksempel 1:}

Vi er givet punkterne $(2,4)$ og $(3,6)$ og bliver bedt om at bestemme forskriften for den rette linje som går igennem disse punkter.

Vi bestemmer først linjens hældning a ved brug af formlen og ved at vælge $(x_1, y_1) = (2,4)$ og $(x_2, y_2) = (3,6)$. Vi beregner a til

\begin{align*}
a = \frac{y_2 - y_1}{x_2 - x_1} = \frac{6 - 4}{3 - 2} = \frac{2}{1} = 2
\end{align*}

Hældningen, a, for den rette linje som går igennem de to punkter er dermed 2. 

Da vi nu har bestemt hældningen af den rette linje, kan vi nu bestemme dens skæring med y-aksen ved brug af formlen

\begin{align*}
b = y_1 - ax_1 = 4 - 2\cdot 2 = 4 - 4 = 0
\end{align*}

Den rette linjes skæring med y aksen er dermed 0. Forskriften for den rette linje som går igennem de 2 punkter er dermed

\begin{align*}
y = 2x + 0 = 2x
\end{align*}


\subsection*{Eksempel 2:}

Vi er givet punkterne $(5,8)$ og $(6,5)$ og bliver bedt om at bestemme forskriften for den rette linje som går igennem disse punkter.

Vi bestemmer først linjens hældning a ved brug af formlen og ved at vælge $(x_1, y_1) = (5,8)$ og $(x_2, y_2) = (6,5)$. Vi beregner a til

\begin{align*}
a = \frac{y_2 - y_1}{x_2 - x_1} = \frac{5 - 8}{6 - 5} = \frac{-3}{1} = -3
\end{align*}

Hældningen, a, for den rette linje som går igennem de to punkter er dermed -3. 

Da vi nu har bestemt hældningen af den rette linje, kan vi nu bestemme dens skæring med y-aksen ved brug af formlen

\begin{align*}
b = y_1 - ax_1 = 8 - (-3)\cdot 5 = 8 + 15 = 23
\end{align*}

Den rette linjes skæring med y aksen er dermed 23. Forskriften for den rette linje som går igennem de 2 punkter er dermed

\begin{align*}
y = -3x + 23
\end{align*}



\subsection*{Opgaver}


\textbf{Opgave 1:}

Bestem forskriften for den rette linje som går igennem punkterne $(2 ,4)$ og $(4 ,10)$

\textbf{Opgave 2:}

Bestem forskriften for den rette linje som går igennem punkterne $(-4 ,4)$ og $(-6 ,8)$

\textbf{Opgave 3:}

Bestem forskriften for den rette linje som går igennem punkterne $(-2 ,10)$ og $(0 ,2)$

\textbf{Opgave 4:}

Bestem forskriften for den rette linje som går igennem punkterne $(-2 ,-2)$ og $(0 ,6)$

\textbf{Opgave 5:}

Bestem forskriften for den rette linje som går igennem punkterne $(2 ,4)$ og $(4 ,-2)$

\textbf{Opgave 6:}

Bestem forskriften for den rette linje som går igennem punkterne $(2 ,4)$ og $(4 ,-4)$

\textbf{Opgave 7:}

Bestem forskriften for den rette linje som går igennem punkterne $(1 ,-4)$ og $(4 ,2)$

\textbf{Opgave 8:}

Bestem forskriften for den rette linje som går igennem punkterne $(2 ,-10)$ og $(4 ,-2)$

\textbf{Opgave 9:}

Bestem forskriften for den rette linje som går igennem punkterne $(-2 ,8)$ og $(4 ,2)$

\textbf{Opgave 10:}

Bestem forskriften for den rette linje som går igennem punkterne $(2 ,2)$ og $(7 ,7)$


\newpage

\subsection*{Facit}


\textbf{Opgave 1:}
\begin{flalign*}
y = 3x -2&&
\end{flalign*}

\textbf{Opgave 2:}
\begin{flalign*}
y = -2x - 4&&
\end{flalign*}

\textbf{Opgave 3:}
\begin{flalign*}
y = -4x + 2&&
\end{flalign*}

\textbf{Opgave 4:}
\begin{flalign*}
y = 4x + 6&&
\end{flalign*}

\textbf{Opgave 5:}
\begin{flalign*}
y = -3x + 10&&
\end{flalign*}

\textbf{Opgave 6:}
\begin{flalign*}
y = -4x + 12&&
\end{flalign*}

\textbf{Opgave 7:}
\begin{flalign*}
y = 2x - 6&&
\end{flalign*} 

\textbf{Opgave 8:}
\begin{flalign*}
y = 4x - 18&&
\end{flalign*}  

\textbf{Opgave 9:}
\begin{flalign*}
y = -x + 6&&
\end{flalign*} 

\textbf{Opgave 10:}
\begin{flalign*}
y = x&&
\end{flalign*}

