\section*{Multiplikation af parenteser}

Når vi ganger 2 parenteser sammen skal vi altid gange alle ledene i den ene parentes med alle ledene i den anden parentes. Et led er typisk adskilt af enten et + eller et -. Vi ganger 2 parenteser sammen som nedenfor

\begin{frm-thm}{Multiplikation af parenteser}
\[(a\tikzmark{a} + b\tikzmark{b}) \cdot (c\tikzmark{c} + d\tikzmark{d}) = ac + ad + bc + bd
\begin{tikzpicture}
[overlay, remember picture, out=80, in = 100, distance = 0.5cm] \draw[->, black, shorten >=8pt, shorten <=3pt] (a.center) to (c.center)
\end{tikzpicture}
\begin{tikzpicture}
[overlay, remember picture, out=60, in = 100, distance = 0.5cm] \draw[->, black, shorten >=8pt, shorten <=3pt] (a.center) to (d.center)
\end{tikzpicture}
\begin{tikzpicture}
[overlay, remember picture, out=280, in = 250, distance = 0.5cm] \draw[->, black, shorten >=8pt, shorten <=3pt] (b.center) to (c.center)
\end{tikzpicture}
\begin{tikzpicture}
[overlay, remember picture, out=280, in = 250, distance = 0.5cm] \draw[->, black, shorten >=8pt, shorten <=3pt] (b.center) to (d.center)
\end{tikzpicture}
\]

\end{frm-thm}


\textbf{Eksempel 1: Gange 2 parenteser sammen}

Beregn følgende udtryk: $(a + 3)\cdot (b - 4)$.

Vi ganger det første led i den første parenten, a, ind på ledene i den anden parentes, b og -4, derefter ganger vi det andet led, 3, ind på alle ledene i den anden parentes og får

\begin{align*}
(a + 3)\cdot (b - 4) = a\cdot b + a \cdot (-4) + 3 \cdot b + 3 \cdot (-4) = ab -4a +3b - 12
\end{align*}


I nogle tilfælde kan vi gange 2 parenteser sammen på en smartere måde. Her kan vi benytte os af de 3 kvadratsætninger som er givet ved


\begin{frm-thm}{Første kvadratsætning}\thlabel{kvad_1}

Lad a og b være 1 tal og 1 bogstav eller 2 forskellige bogstaver så gælder det at

\[(a + b)^2 = a^2 + b^2 + 2ab\]
\end{frm-thm}

\begin{frm-thm}{Anden kvadratsætning}\thlabel{kvad_2}

Lad a og b være 1 tal og 1 bogstav eller 2 forskellige bogstaver så gælder det at

\[(a - b)^2 = a^2 + b^2 - 2ab\]
\end{frm-thm}

\begin{frm-thm}{Tredje kvadratsætning}\thlabel{kvad_3}

Lad a og b være 1 tal og 1 bogstav eller 2 forskellige bogstaver så gælder det at

\[(a + b)\cdot (a - b) = a^2 - b^2\]
\end{frm-thm}

Kvadratsætningerne kan kun bruges på parenteser som indeholder de samme led. Ledene må dog godt være adskilt af både + og -.

For at snakke om i hvilke tilfælde vi benytter os af kvadratsætningerne er det lettest at snakke ud fra en række eksempler.

\subsubsection*{Eksempel 2: Brug af første kvadratsætning}
Hvis vi i en opgave støder på følgende parentes $(x + 3)^2$ eller $(x+3)\cdot (x+3)$ kan vi her benytte os af første kvadratsætning til at ophæve parentesen. Vi kan bruge første kvadratsætning så længe parentesen indeholder 2 led adsklit af et plus hvor de 2 led enten består at 1 tal og 1 bogstav eller 2 forskellige bogstaver. I vores tilfælde består de 2 led af 1 bogstav $x$ og et tal $3$. Bruger vi første kvadratsætning får vi

\begin{align*}
(x + 3)^2 = x^2 + 3^2 + 2\cdot x\cdot 3 = x^2 + 9 + 6x
\end{align*}

Hvis vi ikke havde brugt første kvadratsætning kunne vi ophæve parentesen på følgende måde

\begin{align*}
(x + 3)^2 = (x + 3) \cdot (x + 3) = x^2 + 3x + 3x + 3^2 = x^2 + 6x + 9
\end{align*}

Da dette både er mere besværgeligt og tager længere tid benytter man i stedet for kvadratsætningerne.

\subsubsection*{Eksempel 3: Brug af anden kvadratsætning}
Hvis vi i en opgave støder på følgende parentes $(x - 3)^2$ kan vi her benytte os af anden kvadratsætning da de 2 led i parentesen er adskilt af et minus. Vi får

\begin{align*}
(x - 3)^2 = x^2 + 3^2 -2\cdot x\cdot 3 = x^2 + 9 - 6x
\end{align*}

\subsubsection*{Eksempel 4: Brug af tredje kvadratsætning}
Hvis vi i en opgave støder på følgende parentes $(x + 3)\cdot(x - 3)$ kan vi her benytte os af den tredje kvadratsætning da de 2 parenteser indeholder de samme led, men i den ene parentes er de adskilt af plus og i den anden af minus. Vi får

\begin{align*}
(x + 3)\cdot (x - 3) = x^2 - 3^2 = x^2 - 9
\end{align*}


I vil nu blive testet i at benytte jer af de rigtige kvadratsætninger, og om en af de 3 kvadratsætninger overhovedet kan bruges i de følgende opgaver. 



\subsection*{Opgaver}


\textbf{Opgave 1:}

Beregn $(3+a) \cdot (5-a)$

\textbf{Opgave 2:}

Beregn $(4+a) \cdot (4+a)$

\textbf{Opgave 3:}

Beregn $(5-a) \cdot (5+a)$

\textbf{Opgave 4:}

Beregn $(3-a) \cdot (3-a)$

\textbf{Opgave 5:}

Beregn $(a + 2) \cdot (a + 2)$

\textbf{Opgave 6:}

Beregn $(a - 3) \cdot (a + 3)$

\textbf{Opgave 7:}

Beregn $(a - 5) \cdot (a - 5)$

\textbf{Opgave 8:}

Beregn $(a + 2) \cdot (a + 4)$





\newpage


\subsection*{Facit}


\textbf{Opgave 1:}
I denne opgave kan ingen af de 3 kvadratsætninger bruges.
\begin{flalign*}
-a^2 + 2a - 15&&
\end{flalign*}

\textbf{Opgave 2:}
I denne opgave kan 1. kvadratsæting bruges.
\begin{flalign*}
a^2 + 8a + 16&&
\end{flalign*}

\textbf{Opgave 3:}
I denne opgave kan 3. kvadratsætning bruges.
\begin{flalign*}
-a^2 + 25&&
\end{flalign*}

\textbf{Opgave 4:}
I denne opgave kan 2. kvadratsætning bruges.
\begin{flalign*}
a^2 -6a + 9&&
\end{flalign*}

\textbf{Opgave 5:}
I denne opgave kan 1. kvadratsætning bruges.
\begin{flalign*}
a^2 + 4a + 4&&
\end{flalign*}

\textbf{Opgave 6:}
I denne opgave kan 3. kvadratsætning bruges.
\begin{flalign*}
a^2 - 9&&
\end{flalign*}

\textbf{Opgave 7:}
I denne opgave kan 2. kvadratsætning bruges.
\begin{flalign*}
a^2 -10a + 25&&
\end{flalign*}

\textbf{Opgave 8:}
I denne opgave kan ingen af de 3 kvadratsætninger bruges.
\begin{flalign*}
a^2 + 6a +8&&
\end{flalign*}
