\section*{Andengradsligninger}

For at løse andengradsligninger skal vi bruge følgende formler

\begin{frm-thm}{Løsning af andengradsligning}\thlabel{anden}

Hvis vi har en andengradsligning på følgende form \hspace*{2cm} $ax^2 + bx + c$

Hvor diskriminanten D beregnes ved \hspace*{4.2cm} $D = b^2 -4ac$

Har andengradsligningen følgende løsninger

\vspace*{4mm}

Hvis $D < 0$ findes der ingen løsninger

Hvis $D = 0$ findes der en løsning \hspace*{4.7cm} $x = \cfrac{-b}{2a}$

Hvis $D > 0$ findes der 2 løsninger \hspace*{4.6cm} $x_1 = \cfrac{-b + \sqrt{D}}{2a}, \hspace*{4mm} x_2 = \cfrac{-b - \sqrt{D}}{2a}$

\end{frm-thm}

Vi vil nu vise 3 eksempler hvor vi løser 3 andengradsligninger der har 0, 1 og 2 løsninger.

\textbf{Eksempel 1: 0 løsninger}

Vi er givet andengradsligningen $x^2 + 1 = 0$.

Vi aflæser først a, b og c værdien ud fra den generelle andengradsligning $ax^2 + bx + c = 0$.

a værdien er den værdi der står foran $x^2$, b værdien er den værdi der står foran $x$ og c værdien er den værdi der står for sig selv. Vi har
\begin{align*}
a &= 1\\
b &= 0\\
c &= 1
\end{align*}

Når vi har aflæst værdierne kan vi beregne diskriminanten D. Diskriminanten viser os hvor mange løsninger vores andengradsligning har.

\begin{align*}
D = b^2 - 4ac = 0^2 - 4 \cdot 1 \cdot 1 = -4
\end{align*}

Da $D < 0$ har vores andengradsligning ingen løsninger.



\textbf{Eksempel 2: 1 løsning} 

Vi er givet andengradsligningen $x^2 - 4x + 4 = 0$.

Vi aflæser først a, b og c værdien ud fra den generelle andengradsligning $ax^2 + bx + c = 0$.

a værdien er den værdi der står foran $x^2$, b værdien er den værdi der står foran $x$ og c værdien er den værdi der står for sig selv. Vi har
\begin{align*}
a &= 1\\
b &= -4\\
c &= 4
\end{align*}

Når vi har aflæst værdierne kan vi beregne diskriminanten D. Diskriminanten viser os hvor mange løsninger vores andengradsligning har.

\begin{align*}
D = b^2 - 4ac = (-4)^2 - 4 \cdot 1 \cdot 4 = 16 - 16 = 0
\end{align*}

Da $D = 0$ har vores andengradsligning 1 løsning givet ved $x = \frac{-b}{2a}$.
Løsningen bliver

\begin{align*}
x = \frac{-b}{2a} =\frac{--4}{2\cdot 1} = \frac{4}{2} = 2
\end{align*}


\textbf{Eksempel 3: 2 løsninger}

Vi er givet andengradsligningen $x^2 - 5x + 6 = 0$.

Vi aflæser først a, b og c værdien ud fra den generelle andengradsligning $ax^2 + bx + c = 0$.

a værdien er den værdi der står foran $x^2$, b værdien er den værdi der står foran $x$ og c værdien er den værdi der står for sig selv. Vi har
\begin{align*}
a &= 1\\
b &= -5\\
c &= 6
\end{align*}

Når vi har aflæst værdierne kan vi beregne diskriminanten D. Diskriminanten viser os hvor mange løsninger vores andengradsligning har.

\begin{align*}
D = b^2 - 4ac = (-5)^2 - 4 \cdot 1 \cdot 6 = 25 - 24 = 1
\end{align*}

Da $D > 0$ har vores andengradsligning 2 løsninger givet ved $x_1 = \frac{-b + \sqrt{D}}{2a}$ og $x_2 = \frac{-b-\sqrt{D}}{2a}$
Løsningerne bliver

\begin{align*}
x_1 &= \frac{-b + \sqrt{D}}{2a} = \frac{--5 + \sqrt{1}}{2\cdot 1} = \frac{5 + 1}{2} = 3\\
x_2 &= \frac{-b - \sqrt{D}}{2a} = \frac{--5 - \sqrt{1}}{2\cdot 1} = \frac{5 - 1}{2} = 2
\end{align*}


\section*{Opgaver}

\textbf{1.} $4x^2 -2x + 5 = 0$

\textbf{2.} $x^2 - 1 = 0$

\textbf{3.} $x^2 - 9 = 0$

\textbf{4.} $x^2 - 3x - 10 = 0$

\textbf{5.} $x^2 - 10x + 21 = 0$

\textbf{6.} $x^2 - 25 = 0$

\textbf{7.} $ x^2 -3x - 28= 0$

\textbf{8.} $x^2 - 9 = 0$

\textbf{9.} $x^2 + x - 72= 0$

\textbf{10.} $x^2 - 16x + 55 = 0$

\newpage

\section*{Facit}

\textbf{1.} Ingen løsning

\textbf{2.} $x = 1\hspace*{2mm} \vee \hspace*{2mm} x = -1$

\textbf{3.} $x = 3\hspace*{2mm} \vee \hspace*{2mm} x = -3$

\textbf{4.} $x = 5\hspace*{2mm} \vee \hspace*{2mm} x = -2$

\textbf{5.} $x = 7\hspace*{2mm} \vee \hspace*{2mm} x = 3$

\textbf{6.} $x = -5\hspace*{2mm} \vee \hspace*{2mm} x = 5$

\textbf{7.} $x = -4\hspace*{2mm} \vee \hspace*{2mm} x = 7$

\textbf{8.} $x = -3\hspace*{2mm} \vee \hspace*{2mm} x = 3$

\textbf{9.} $x = 8\hspace*{2mm} \vee \hspace*{2mm} x = -9$

\textbf{10.} $x = 11\hspace*{2mm} \vee \hspace*{2mm} x = 5$
