\section*{Relationer i en vilkårlig trekant}

I den vilkårlige trekant gælder både sinusrelationerne og cosinusrelationerne. De kan ses nedenfor


\begin{frm-thm}{Sinusrelationerne}

I den vilkårlige trekant gælder følgende relationer

\[\frac{a}{\sin(A)} = \frac{b}{\sin(B)} = \frac{c}{\sin(C)}\]

\end{frm-thm}

\begin{frm-thm}{Cosinusrelationerne}

I den vilkårlige trekant gælder følgende 3 relationer

\[a^2 = b^2 + c^2 - 2bc\cos(A) \hspace*{2cm} A = \cos^{-1}\left(\frac{b^2 + c^2 - a^2}{2bc}\right)\]

\[b^2 = a^2 + c^2 - 2ac\cos(B) \hspace*{2cm} B = \cos^{-1}\left(\frac{a^2 + c^2 - b^2}{2ac}\right)\]

\[c^2 = a^2 + b^2 - 2ab\cos(C) \hspace*{2cm} C = \cos^{-1}\left(\frac{a^2 + b^2 - c^2}{2ab}\right)\]

\end{frm-thm}

Vi kan bruge sinusrelationerne til at bestemme alle vinkler og sidelængder i en vilkårlig trekant så længe vi kender 1 side og 2 vinkler eller 2 sider og 1 vinkel.

Vi kan bruge cosinusrelationerne til at bestemme den manglende sideængde ud fra de 2 andre sidelængder og 1 vinkel eller bestemme alle vinkler i trekanten hvis alle siderne er kendte.

Vi vil nu vise hvordan vi bruger cosinusrelationerne og sinusrelationerne.

\textbf{Eksempel 1: Brug af cosinusrelationerne}



\textbf{Eksempel 2: Brug af sinusrelationerne}



\subsection*{Opgaver}

\textbf{Opgave 1:}

\textbf{Opgave 2:}

\textbf{Opgave 3:}

\textbf{Opgave 4:}

\textbf{Opgave 5:}



\newpage


\subsection*{Facit}

\textbf{Opgave 1:}

\textbf{Opgave 2:}

\textbf{Opgave 3:}

\textbf{Opgave 4:}

\textbf{Opgave 5:}
