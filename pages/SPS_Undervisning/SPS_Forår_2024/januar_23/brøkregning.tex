\section{Brøkregning}

\subsection{Brøkregneregler}

Det første emne vi kommer til at gennemgå er brøkregneregler. Vi skal lære addition af brøker, subtraktion af brøker, hvordan vi ganger og dividerer brøker og til sidst hvordan vi forkorter brøker. 

En brøk består af 2 tal, det øverste tal kaldes for tælleren og det nederste tal for nævneren. Når vi adderer 2 brøker gælder følgende regler

\begin{frm-thm}{Addition og subtraktion af brøker}\thlabel{add_sub}


Hvis 2 brøker har samme nævner b er additionen og subtraktionen af de 2 brøker givet ved
\[ \frac{a}{b} + \frac{c}{b} = \frac{a + c}{b}   \]
\[\frac{a}{b} - \frac{c}{b} = \frac{a - c}{b}\]

Hvis 2 brøker ikke har samme nævner er deres addition givet ved
\[ \frac{a}{b} + \frac{c}{d} = \frac{a\cdot d + c \cdot b}{b \cdot d} \]
\[\frac{a}{b} - \frac{c}{d} = \frac{a\cdot d - c\cdot b}{b \cdot d} \]
\end{frm-thm}

Vi vil nu gennemgå en række eksempler på hvordan vi laver addition og subtraktion af 2 brøker.

\subsubsection*{Eksempel 1: Addition med ens nævnere}

Vi får givet følgende brøker $\frac{2}{3}$ og $\frac{5}{3}$ og bliver bedt om at bestemme $\frac{2}{3} + \frac{5}{3}$. Da de 2 brøker har fælles nævner siger theorem \ref{add_sub} at vi skal gøre følgende
\begin{align*}
\frac{2}{3} + \frac{5}{3} = \frac{2 + 5}{3} = \frac{7}{3}
\end{align*}

Resultatet af additionen er dermed $\frac{7}{3}$.

\subsubsection*{Eksempel 2: Subtraktion med ens nævnere}

Vi får givet følgende brøker $\frac{2}{3}$ og $\frac{1}{3}$ og bliver bedt om at bestemme $\frac{2}{3} - \frac{1}{3}$. Da de 2 brøker har fælles nævnere siger theorem \ref{add_sub} at vi skal gøre følgende

\begin{align*}
\frac{2}{3} - \frac{1}{3} = \frac{2 - 1}{3} = \frac{1}{3}
\end{align*}

Resultatet af subtraktionen er dermed $\frac{1}{3}$

\subsubsection*{Eksempel 3: Addition med forskellige nævnere}

Vi får givet følgende brøker $\frac{2}{3}$ og $\frac{4}{5}$ og bliver bedt om at bestemme $\frac{2}{3} + \frac{4}{5}$. Da de 2 brøker ikke har fælles nævnere siger theorem \ref{add_sub} at vi skal gøre følgende

\begin{align*}
\frac{2}{3} + \frac{4}{5} = \frac{2\cdot 5 + 4\cdot 3}{3\cdot 5} = \frac{10 + 12}{15} = \frac{22}{15}
\end{align*}

Resultatet af additionen er dermed $\frac{22}{15}$

\subsubsection*{Eksempel 4: Subtraktion med forskellige nævnere}

Vi får givet følgende brøker $\frac{2}{3}$ og $\frac{4}{5}$ og bliver bedt om at bestemme $\frac{2}{3} - \frac{4}{5}$. Da de 2 brøker ikke har fælles nævnere siger theorem \ref{add_sub} at vi skal gøre følgende

\begin{align*}
\frac{2}{3} - \frac{4}{5} = \frac{2\cdot 5 - 4\cdot 3}{3\cdot 5} = \frac{10 - 12}{15} = \frac{-2}{15} = -\frac{2}{15}
\end{align*}


Når vi ganger et tal på en brøk, ganger 2 brøker sammen eller dividerer 2 brøker gælder følgende regler

\begin{frm-thm}{Multiplikation og division af brøker}\thlabel{mult_div}
Hvis vi ganger tallet a ind på en brøk ganger vi a ind i tælleren
\[a\cdot \frac{b}{c} = \frac{a\cdot b}{c}\]

Ganger vi 2 brøker med hinanden ganger vi deres tællere og nævnere sammen
\[\frac{a}{b} \cdot \frac{c}{d} = \frac{a\cdot c}{b\cdot d} \]

Dividerer vi 1 brøk med en anden kan vi i stedet gange med den omvendte brøk (dvs vi bytter om på tælleren og nævneren i den ene brøk)

\[\frac{a}{b} : \frac{c}{d} = \frac{a}{b} \cdot \frac{d}{c}\]
\end{frm-thm}

Vi vil nu gennemgå en række eksempler på hvordan vi kan bruge de ovenstående regler

\subsubsection*{Eksempel 5: Multiplikation af konstant og brøk}

Vi får givet konstanten $4$ og brøken $\frac{3}{5}$ og bliver bedt om at bestemme $4\cdot \frac{3}{5}$. Når vi ganger et tal på en brøk siger theorem \ref{mult_div} at vi skal gøre følgende

\begin{align*}
4\cdot \frac{3}{5} = \frac{4\cdot 3}{5} = \frac{12}{5}
\end{align*}

Resultatet af multiplikationen er dermed $\frac{12}{5}$


\subsubsection*{Eksempel 6: Multiplikation af 2 brøker}

Vi får givet de 2 brøker $\frac{2}{3}$ og $\frac{4}{5}$ og bliver bedt om at bestemme $\frac{2}{3} \cdot \frac{4}{5}$. Når vi ganger 2 brøker sammen siger theorem \ref{mult_div} at vi skal gøre følgende

\begin{align*}
\frac{2}{3} \cdot \frac{4}{5} = \frac{2\cdot 4}{3\cdot 5} = \frac{8}{15}
\end{align*}

Resultatet af multiplikationen er dermed $\frac{8}{15}$

\subsubsection*{Eksempel 7: Divison af 2 brøker}

Vi får givet de 2 brøker $\frac{2}{3}$ og $\frac{4}{5}$ og bliver bedt om at bestemme $\frac{2}{3} : \frac{4}{5}$. Når vi dividerer en brøk med en anden siger theorem \ref{mult_div} at vi skal gøre følgende

\begin{align*}
\frac{2}{3} : \frac{4}{5} = \frac{2}{3} \cdot \frac{5}{4} = \frac{2\cdot 5}{3\cdot 4} = \frac{10}{12}
\end{align*}

Resultatet af divisionen er dermed $\frac{10}{12}$

Vi kigger nu på følgende opgaver

\subsection{Opgave: Addition af brøker med fælles nævner}
Beregn $\frac{1}{2} + \frac{3}{2}$.

Da brøkerne har fælles nævner kan vi ifølge Theorem \ref{add_sub} lægge tællerne sammen og vi får

\begin{align*}
\frac{1}{2} + \frac{3}{2} = \frac{1 + 3}{2} = \frac{4}{2} = 2
\end{align*}

Resultatet af additionen er dermed $2$.

\subsection{Opgave: Addition af brøker med forskellige nævnere}
Beregn $\frac{1}{3} + \frac{1}{6}$.

Da brøkerne ikke har fælles nævner, skal vi først forlænge den ene af brøkerne så de får fælles nævner. Da den først brøk har nævneren 3 og den anden brøk nævneren 6, kan brøkerne få fælles nævneren 6 ved at gange 2 på tælleren og nævneren af den første brøk og derefter lægge brøkerne sammen. Vi får dermed

\begin{align*}
\frac{1}{3} + \frac{1}{6} = \frac{2\cdot 1}{2\cdot 3} + \frac{1}{6} = \frac{2}{6} + \frac{1}{6} = \frac{2 + 1}{6} = \frac{3}{6} = \frac{1}{2}
\end{align*}

Resultatet af additionen er dermed $\frac{1}{2}$.

\subsection{Opgave: Subtraktion af bøker med forskellige nævnere}
Beregn $\frac{1}{2} - \frac{1}{4}$.

Da brøkerne ikke har fælles nævnere skal vi forlænge den ene af brøkerne så de får fælles nævner. Da den første brøk har nævneren 2 og den anden nævneren 4, kan brøkerne få fælles nævneren 4 ved at gange 2 på tælleren og nævneren af den første brøk og derefter trække brøkerne sammen. Vi får dermed

\begin{align*}
\frac{1}{2} - \frac{1}{4} = \frac{2\cdot 1}{2\cdot 2} - \frac{1}{4} = \frac{2}{4} - \frac{1}{4} = \frac{2 - 1}{4} = \frac{1}{4}
\end{align*}

Resultatet af subtraktionen er dermed $\frac{1}{4}$.

\subsection{Opgave: Subtraktion af brøker med forskellige nævnere}
Beregn $\frac{3}{4} - \frac{1}{2}$.

Da brøkerne ikke har fælles nævnere skal vi forlænge den ene af brøkerne så de får fælles nævner. Da den første brøk har nævneren 4 og den anden nævneren 2, kan brøkerne få fælles nævneren 4 ved at gange 2 på tælleren og nævneren af den anden brøk og derefter trække brøkerne sammen. Vi får dermed

\begin{align*}
\frac{3}{4} - \frac{1}{2} = \frac{3}{4} - \frac{2\cdot 1}{2\cdot 2} = \frac{3}{4} - \frac{2}{4} = \frac{3-2}{4} = \frac{1}{4}
\end{align*}

Resultatet af subtraktionen er dermed $\frac{1}{4}$.

\subsection{Opgave: Multiplikation af brøker}
Beregn $\frac{2}{1}\cdot \frac{1}{2}$.

For at gange 2 brøker sammen siger Theorem \ref{mult_div} at vi ganger tællerne sammen og nævnerne sammen. Vi får dermed

\begin{align*}
\frac{2}{1}\cdot \frac{1}{2} = \frac{2\cdot 1}{1\cdot 2} = \frac{2}{2} = 1
\end{align*}

Resultatet af multiplilationen er dermed $1$.

\subsection{Opgave: Multiplikation af brøker}
Beregn $\frac{1}{4}\cdot \frac{1}{3}$.

For at gange 2 brøker sammen siger Theorem \ref{mult_div} at vi ganger tællerne sammen og nævnerne sammen. Vi får dermed

\begin{align*}
\frac{1}{4}\cdot \frac{1}{3} = \frac{1\cdot 1}{4\cdot 3} = \frac{1}{12}
\end{align*}

Resultatet af multiplikationen er dermed $\frac{1}{12}$.

\subsection{Opgave: Division af brøker}
Beregn $\frac{1}{4}:\frac{1}{3}$.

For at dividere 1 brøk med en anden brøk siger Theorem \ref{mult_div} at vi skal gange med den omvendte brøk. Vi får dermed

\begin{align*}
\frac{1}{4}:\frac{1}{3} = \frac{1}{4} \cdot \frac{3}{1} = \frac{1\cdot 3}{4\cdot 1} = \frac{3}{4}
\end{align*}

Resultatet af divisionen er dermed $\frac{3}{4}$.

