\section{Simple ligninger}

I denne opgave skal vi løse simple ligninger. For at løse en simpel ligning flytter vi alle tallene over på den ene side af lighedstegnet og alle x'erne over på den anden side. Til sidst finder vi en værdi for x som er løsningen på vores ligning. Når vå har fundet værdien for x kan vi teste om det er den rigtige værdi ved at indsætte værdien på x's plads i ligningen of tjekke om begge sider af lighedstegnet er ens. Nedenfor er der 3 opgaver med simple ligninger.


\subsection{Opgave}
Løs ligningen $3 + x = 7$.

Vi løser ligningen ved at flytte tallene over på den højre side af lighedstegnet. Vi får

\begin{align*}
3 + x &= 7\\
\Updownarrow \hspace*{14mm} &\\
3 + x - 3 &= 7 - 3 && \text{Trækker 3 fra på begge sider}\\ 
\Updownarrow \hspace*{14mm} &\\
x &= 4
\end{align*}

Når vi løser ligningen får vi at $x = 4$. Vi tjekkerom resultatet er korrekt ved at indsætte $x = 4$ i ligningen

\begin{align*}
3 + x &= 7\\
\Updownarrow \hspace*{8mm} &\\
3 + 4 &= 7 && \text{Indsætter} \hspace*{1mm} x = 4 \hspace*{1mm} \text{på x's plads}\\
\Updownarrow \hspace*{8mm} &\\
7 &= 7
\end{align*}

Da højre og venstre siden af lighedstegnet er ens har vi dermed løst ligningen korrekt. Dette check er ikke nødvendigt, men hvis man er i tvivl om man har løst en ligning korrekt kan man altid tjekke efter ligesom vi har gjort ovenfor.




\subsection{Opgave}
Løs ligningen $4 + 2x = 12$

Vi løser ligningen ved at flytte tallene over på den højre side af lighedstegnet. Vi får

\begin{align*}
4 + 2x &= 12\\
\Updownarrow \hspace*{16mm} &\\
4 + 2x - 4 &= 12 - 4 && \text{Trækker 4 fra på begge sider}\\
\Updownarrow \hspace*{16mm} &\\
2x &= 8\\
\Updownarrow \hspace*{16mm} &\\
\frac{2x}{2} &= \frac{8}{2} && \text{Dividerer med 2 på begge sider så x står alene}\\
\Updownarrow \hspace*{16mm} &\\
x &= 4
\end{align*}

Når vi løser ligningen får vi at $x = 4$. Vi tjekker om resultatet er korrekt ved at indsætte $x = 4$ i ligningen

\begin{align*}
4 + 2x &= 12\\
\Updownarrow \hspace*{12mm} &\\
4 + 2\cdot 4 &= 12 && \text{Indsætter} \hspace*{1mm} x = 4 \hspace*{1mm} \text{på x's plads}\\
\Updownarrow \hspace*{12mm} &\\
4 + 8 &= 12\\
\Updownarrow \hspace*{12mm} &\\
12 &= 12
\end{align*}

Da højre og venstre siden af lighedstegnet er ens har vi dermed løst ligningen korrekt. Dette check er ikke nødvendigt, men hvis man er i tvivl om man har løst en ligning korrekt kan man altid tjekke efter ligesom vi har gjort ovenfor.

\subsection{Opgave}
Løs ligningen $3x - 2 = 7$

Vi løser ligningen ved at flytte tallene over på den højre side af lighedstegnet. Vi får

\begin{align*}
3x - 2 &= 7\\
\Updownarrow \hspace*{16mm} &\\
3x - 2 + 2 &= 7 + 2 && \text{Lægger 2 til på begge sider}\\
\Updownarrow \hspace*{16mm} &\\
3x &= 9\\
\Updownarrow \hspace*{16mm} &\\
\frac{3x}{3} &= \frac{9}{3} && \text{Dividierer med 3 på begge sider så x står alene}\\
\Updownarrow \hspace*{16mm} &\\
x &= 3
\end{align*}

Når vi løser ligningen får vi at $x = 3$. Vi tjekker om resultatet er korrekt ved at indsætte $x = 3$ i ligningen

\begin{align*}
3x - 2 &= 7\\
\Updownarrow \hspace*{12mm} &\\
3\cdot 3 - 2 &= 7 && \text{Indsætter} \hspace*{1mm} x = 3 \hspace*{1mm} \text{på x's plads}\\
\Updownarrow \hspace*{12mm} &\\
9 - 2 &= 7\\
\Updownarrow \hspace*{12mm} &\\
7 &= 7
\end{align*}

Da højre og venstre siden af lighedstegnet er ens har vi dermed løst ligningen korrekt. Dette check er ikke nødvendigt, men hvis man er i tvivl om man har løst en ligning korrekt kan man altid tjekke efter ligesom vi har gjort ovenfor.